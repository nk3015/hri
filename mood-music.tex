\documentclass{article}
\title{Mood-Based Music DJ}
\author{P. Girish}
\date{October 2024}
\usepackage{biblatex}
\begin{document}

\maketitle

\section{Objective}
 \subsection{Scenario}
Cozmo can identify the user’s mood and play music to suit it.

\subsection{Tasks}
Cozmo can detect a user's emotional state via facial expressions (e.g., happiness, sadness) using a facial recognition API or mood-sensing APIs like Affectiva.
Based on the detected mood, Cozmo can play a song from a connected music service (Spotify API or YouTube Music API).
Reaction: Cozmo dances and reacts to the music genre (e.g., head bobbing for a happy song, sad movements for a slow ballad).

\section{Introduction} 
The Cozmo robot, developed by Anki and introduced in 2016, is a compact and engaging social robot designed to interact with users in a playful and educational manner. With physical dimensions of approximately 10.5 cm in height and 9.5 cm in width, Cozmo is small enough to fit in the palm of a hand, making it an ideal companion for both children and adults. The robot is powered by a rechargeable battery that provides up to 30 minutes of active playtime, allowing for extended interaction sessions. Cozmo features a range of expressive emote presets, with over 20 different facial expressions and sounds, enabling it to convey emotions effectively and enhance user engagement. This project aims to explore and implement interactive behaviors for Cozmo in a novel play setting, utilizing its capabilities to foster meaningful human-robot interactions in various contexts.

\section{Literature Review}
\section{Methodology}
\section{Data Analysis and Interpretation}
\section{Findings and Recommendations}
\section{Conclusion}

\end{document}